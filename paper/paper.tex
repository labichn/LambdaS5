\documentclass[preprint,9pt]{sigplanconf} %{onecol}
\usepackage{amsfonts,amsmath,amsthm}
\usepackage{pxfonts}
\usepackage{natbib}
\usepackage{bm}
\usepackage{balance}
\usepackage{wrapfig}
\usepackage{comment}
\usepackage[T1]{fontenc}

\newtheorem{mdef}{Definition}
\newtheorem{mlem}{Lemma}
\newtheorem{mtheo}{Theorem}

\usepackage{xcolor}
\newcommand\mtodo[1]{\textcolor{red}{\text{<TODO: \MakeUppercase{#1}>}}}

% stolen from Ian
\newcommand{\superscript}[1]{\ensuremath{^{#1}}}
\newcommand{\subscript}[1]{\ensuremath{_{#1}}}
\newcommand{\tuple}[3][\ ]{{\tt #2}{#1}({#3})}

% λS5 expressions and values
\newcommand{\mexp}{e}
\newcommand{\mloc}{l}
\newcommand{\mlit}{lit}
\newcommand{\mval}{v}
\newcommand{\mvexp}{ve}
\newcommand{\mvar}{x}
\newcommand{\mpexp}{pe}
\newcommand{\maexp}{ae}
\newcommand{\moattr}{oa}
\newcommand{\mpattr}{pa}
\newcommand{\moloc}{r}
\newcommand{\mopone}{op1}
\newcommand{\moptwo}{op2}
\newcommand{\mstr}{s}
\newcommand{\mnum}{n}

% spaces
\newcommand{\mkeys}{\ensuremath{\mathcal{K}}}
\newcommand{\mhashs}{\ensuremath{\mathcal{H}}}
\newcommand{\mvals}{\ensuremath{\mathcal{V}}}
\newcommand{\mpset}{\ensuremath{\mathcal{P}}}

% machine components
%\newcommand{\ev}[2][\ ]{\tuple[#1]{ev}{#2}}
%\newcommand{\co}[2][\ ]{\tuple[#1]{co}{#2}}
%\newcommand{\ap}[2][\ ]{\tuple[#1]{ap}{#2}}
%\newcommand{\ans}[2][\ ]{\tuple[#1]{ans}{#2}}
%\newcommand{\ap}[2][\ ]{\tuple[#1]{ap}{#2}}
%\newcommand{\clos}[1]{\tuple{clos}{#1}}


% need proper command:
\newcommand{\mmonjoin}{MONJOIN}


\begin{document}

\conferenceinfo{WXYZ '05}{date, City.}
\copyrightyear{2005}
\copyrightdata{[to be supplied]}

%\titlebanner{banner above paper title}
%\preprintfooter{short description of paper}

\title{Analyzing JavaScript: The Bad Parts}

\authorinfo{Nicholas Labich}
           {Northeastern University}
           {labichn@ccs.neu.edu}
\authorinfo{Shuying Liang}
           {University of Utah}
           {liansy@cs.utah.edu}
\authorinfo{Matthew Might}
           {University of Utah}
           {might@cs.utah.edu}
\authorinfo{David Van Horn}
           {Northeastern University}
           {dvanhorn@ccs.neu.edu}

\maketitle

\mtodo{Something pithy about JS: the good parts in abstract or
  intro... might be a good task for Might.}

\begin{abstract}
JavaScript is the ubiquitous, powerful, and flagrantly-dynamic lingua
franca of the web. Though the frequent object of current research, to
date attempts at static analysis tackle calculated subsets of the
language, eliding (thereby going unsound for) the darker corners:
property and object descriptors, getters and setters, strict mode,
and eval. Leveraging $\lambda$S5, a core semantics encapsulating the
ECMAScript 5.1 standard, we detail the approach and implementation of
a \emph{sound} abstract interpretation of full, modern JavaScript.
\end{abstract}

\section{Introduction}
JavaScript has clear domain over today's web programming. It has been
the tool of choice on the client-side for some time. The popularity of
Node.js and other server-side JavaScript solutions is growing. It is
used as a cross-compilation target from C, Java, OCaml, Perl, Scala,
and many others. JavaScript's expanding community continues to use
this weakly-typed, dynamic scripting language for large-scale projects
in both client- and server-side settings. The need for disciplined and
sound methods of reasoned analysis grows with its popularity.

\mtodo{This paragraph seems more related-work-esque}
There has been a significant amount of quality research wrt. analysis
of JavaScript. Much of this research has focused on security and
correctness~\cite{Richards:dynamism}. As \emph{the} front-end language
of the web, JavaScript is a prime target for attack, and there has
been a great deal of interest in analysis as tool to improve
security~\cite{chugh2009staged, Guha:2009:USA:1526709.1526785,
  maffeis2009isolating, vogt2007cross,
  Yu:2007:JIB:1190216.1190252}. As a dynamic scripting language, much
work has focused on error detection by augmenting JavaScript with a
static type system~\cite{anderson2006type, anderson2003babyj,
  anderson2005type, heidegger2009recency, heidegger2009recency,
  hackett2012fast}. \mtodo{No leading as in this sentence:} As detailed
by Richards~\cite{Richards:dynamism}, these efforts operate on subsets
of the language, often ignoring many of the nuances, pecularities, and
dark corners: the global object, prototypical inheritence, prototype
dynamism, property addition/deletetion, and eval.  Moreover, all of
this work targets pre-ES5.1 JavaScript. It is these aspects of
JavaScript's informally specified semantics that we (jocularly) refer
to as the \emph{bad parts}.

The lack of a formal semantics has been one of the largest hurdles
preventing a disciplined analysis of JavaScript\footnote{The use of
  'JavaScript' hereafter denotes EMCAScript 5.1
  (ES5)~\cite{ecma2009ecma}; exceptions will be noted
  explicitly.}. The first full and formal specification of the
ECMAScript 3rd edition (ES3) was released in
2008~\cite{Ramalingam:PLS}, nearly ten years after the specification's
publication. Guha et al. took a different
approach~\cite{Guha:2010:EJ:1883978.1883988}, implementing a core
language, $\lambda$JS, that when paired with a desugaring function
from ES3 to $\lambda$JS, effectively and accurately models the ES3
semantics, sans eval. Politz et
al.~\cite{Politz:2012:TSG:2480360.2384579} continued with
$\lambda$JS's approach with $\lambda$S5. $\lambda$S5 formalizes and
implementes a core syntax and semantics that, paired with a desugaring
function, models all of JavaScript's semantics, including object and
property descriptors, getters and setters, strict mode, and
eval. Details of the semantics and desugaring will be explicated where
they play a pivital role in analysis; full exposition and design
rationale can be found in ~\emph{A tested semantics for getters,
  setters, and eval in
  JavaScript}~\cite{Politz:2012:TSG:2480360.2384579}.

Targeting $\lambda$S5 is not without disadvantages. The desugared code
is larger than its sugared counterpart, translating what could be
logic internal to the analyzer into expressions which must be
evaluated. Also, much of the global JavaScript functionality is
implemented in an initial environment written directly in $\lambda$S5,
which complicates analysis \mtodo{Point to later discussion}. Despite
these obstacles, $\lambda$S5 is far more amenable to description
by a reduction semantics and to evaluation by abstract machine,
enabling principled and \emph{sound} abstraction. As the only formal
semantics for full, modern JavaScript, we focus our aim on the
abstract interpretation of $\lambda$S5.

\paragraph{Contributions}
\mtodo{goto Might about properly articulating these}
\begin{itemize}
\item Details of the abstractions and implementation techniques that
  entail a tractably computable analysis of JavaScript, with careful
  attention paid to \emph{the bad parts}.
\item A sound and provably correct abstract interpreter of full,
  modern JavaScript through $\lambda$S5.
\end{itemize}

\paragraph{Background}
We assume some familiarity with reduction semantics and abstract
machines~\cite{felleisen1986control,felleisen2009semantics}. The leap
from abstract machine to abstract interpreter follows the methodology
of Van Horn \& Might's \emph{Abstracting Abstract Machines} and later
work on optimization of such analyzers~\cite{van2010abstracting,
  DBLP:journals/corr/abs-1211-3722}.

\section{The Core of JavaScript}
Brendan Eich, per his personal
blog\footnote{https://brendaneich.com/2008/04/popularity/}, ``was
recruited to Netscape with the promise of `doing Scheme' in the
browser''. His efforts birthed the original JavaScript. It is
unsurprising that the resultant language is susceptible to desugaring
into a functional core language, though, as detailed by Politz et
al.~\cite{Politz:2012:TSG:2480360.2384579} and as we will discuss
later \mtodo{Point to later discussion}, the process is far from
trivial. We take the desugaring process to be a black-box for now
(and later as a parameter to our analyzer \mtodo{Point to later
  discussion}).  After desugaring we are left with $\lambda$S5 (figure
~\ref{fig:syntax}).

What remains is akin to a curly-braced member of Landin's ISWIM
family~\cite{landin1966next}, extended with a full compliment of
values, set!-like side effects, labeled and exceptional control-flow,
eval, and a map-like object system. A lexically-scoped, functional
language, $\lambda$S5's semantics can easily be described as the
reduction relation of an abstract machine~\cite{felleisen1986control,
  felleisen2009semantics}. Before we dive into the machine's
construction and abstraction, we should take a closer look at the
main difficulties.

\begin{figure}
%\small
\[
\begin{array}{ccll}
\mvar&:=&\multicolumn{2}{l}{variables}\\
\mnum&:=&\multicolumn{2}{l}{numeric\ values}\\
\mstr&:=&\multicolumn{2}{l}{string\ values}\\
\moloc&:=&\multicolumn{2}{l}{object\ references}\\
\mlit&:=&\multicolumn{2}{l}{\mnum\ |\ \mstr\ |\ \moloc\ |\ {\bf{null}}\ |\ {\bf{undefined}}\ |\ {\bf{true}}\ |\ {\bf{false}}}\\
\mvexp&:=&\multicolumn{2}{l}{\mlit\ |\ {\bf{func}}(\mvar,\dots)\ \{\ \mexp\ \}}\\\\
%\multicolumn{4}{l}{\text{Attribute names}}\\
\moattr&:=&\multicolumn{2}{l}{{\bf{class}}\ |\ {\bf{extensible}}\ |\ {\bf{proto}}\ |\ {\bf{code}}\ |\ {\bf{primval}}}\\
\mpattr&:=&\multicolumn{2}{l}{{\bf{writable}}\ |\ {\bf{config}}\ |\ {\bf{value}}\ |\ {\bf{enum}}}\\\\
%\multicolumn{4}{l}{\text{Property expressions}}\\
\mpexp&:=&\multicolumn{2}{l}{\text{[}{\bf{config}}:\mexp,\ {\bf{enum}}:\mexp,\ {\bf{value}}:\mexp,\ {\bf{writable}}:\mexp\text{]}}\\
&|&\multicolumn{2}{l}{\text{[}{\bf{config}}:\mexp,\ {\bf{enum}}:\mexp,\ {\bf{get}}:\mexp,\ {\bf{set}}:\mexp\text{]}}\\\\
%\multicolumn{4}{l}{\text{Object attributes expressions}}\\
\maexp&:=&\multicolumn{2}{l}{\text{[}{\bf{class}}:\mexp,{\bf{extensible}}:\mexp,{\bf{proto}}:\mexp,{\bf{code}}:\mexp,{\bf{primval}}:\mexp\text{]}}\\\\
%\multicolumn{4}{l}{\text{Primops}}\\
\mopone&:=&\multicolumn{2}{l}{\text{string->num}\ |\ {\bf{typeof}}\ |\ \text{log}\ |\ \text{prim->bool}\ |\ \dots}\\
\moptwo&:=&\multicolumn{2}{l}{\text{string-append}\ |\ \text{+}\ |\ \div\ |\ \text{>}\ |\ \dots}\\\\
%\multicolumn{4}{l}{\text{Expressions}}\\
\mexp&:=&\multicolumn{2}{l}{\mvexp\ |\ \mvar\ |\ \mvar\ :=\ \mexp\ |\ \mopone(\mexp)\ |\ \moptwo(\mexp, \mexp)\ |\ \mexp(\mexp,\dots)}\\
&|&\multicolumn{2}{l}{\mexp;\mexp\ |\ {\bf{let}}\ \ (\mvar\ \text{=}\ \mexp)\ \mexp\ |\ {\bf{if}}\ \ (\mexp)\ \{\ \mexp\ \}\ {\bf{else}}\ \ \{\ \mexp\ \}}\\
&|&\multicolumn{2}{l}{{\bf{label}}:\ \mvar\ \mexp\ |\ \ {\bf{break}}\ \ \mvar\ \mexp\ |\ \ {\bf{err}}\ \ \mvexp}\\
&|&\multicolumn{2}{l}{{\bf{try}}\ \ \mexp\ \ {\bf{catch}}\ \ \mvar\ \mexp\ |\ {\bf{try}}\ \ \mexp\ \ {\bf{finally}}\ \ \mexp\ |\ \ {\bf{throw}}\ \ \mexp}\\
&|&\multicolumn{2}{l}{{\bf{eval}}(\mexp,\mexp)}\\
&|&\{\maexp\ \ \mstr:\mpexp,\dots\}&object\ literals\\
&|&\mexp\text{[}\mexp\text{]}\ |\ \mexp\text{[}\mexp\text{=}\mexp\text{]}\ |\ \mexp\text{[}{\bf{delete}}\ \mexp\text{]}&properties\\
&|&{\bf{props}}(\mexp)&property\ names\\
&|&\mexp\text{[<}\moattr\text{>]}\ |\ \mexp\text{[<}\moattr\text{>}\text{=}\mexp\text{]}&object\ attributes\\
&|&\mexp\text{[}\mexp\text{<}\mpattr\text{>]}\ |\ \mexp\text{[}\mexp\text{<}\mpattr\text{>=}\mexp\text{]}&property\ attributes\\
\end{array}
\]
\caption{Syntax of $\lambda$S5}
\label{fig:syntax}
\end{figure}

\section{The \emph{Bad Parts}}

\subsection{Scope and the global object}

$\lambda$S5 is lexically-scoped, which makes quick work of
JavaScript's Escherian scoping behavior. Consider these simple
examples\footnote{Borrowed from the third author's blog:
  http://matt.might.net /articles/javascript-warts/}:
\begin{verbatim}
x;  // ReferenceError: x is undefined
\end{verbatim}
Accessing an unbound variable raises an exception, as
expected. However:
\begin{verbatim}
x;  // No problem! x is declared in this scope.
if (false) {
  var x;
}
\end{verbatim}
This code merely returns the value \emph{undefined}, which denotes
that the the variable exists but has not been assigned. JavaScript's
variables are properties in a singleton global object; variable
definitions, including those in code that will never run, are hoisted
into that global object. $\lambda$S5 handles JavaScript's global
object and its unusual scoping rules by converting variable accesses
to field accesses in a let-bound object \texttt{\%context}. The
desugarer simply ensures the proper context binding exists where it is
needed:
\begin{verbatim}
  {let (%context = %nonstrictContext)
    %defineGlobalVar(%context, "x");
    {let (#strict = false)
      %context["x",
               {[#proto: null,
                 #class: "Object",
                 #extensible: true,]}];
      if (%ToBoolean(false)) { undefined }
      else { undefined }}}
\end{verbatim}

$\lambda$S5's initial environment includes JavaScript's global
objects, such as \texttt{console}, \texttt{Object}, et cetera. It also
includes objects and functions inserted as part of the desugaring
process, including the \texttt{\%context} object and an
\emph{\%ErrorDispatch} function. The initial environment is
implemented directly in \textasciitilde5,000 lines of $\lambda$S5,
which complicates analysis. Consider the following simple program:
\begin{verbatim}
  var x = 1;
  var y = 2;
  x;
\end{verbatim}
which desugars to the following $\lambda$S5 expression:
\begin{verbatim}
  {let (%context = %nonstrictContext)
    %defineGlobalVar(%context, "y");
    %defineGlobalVar(%context, "x");
    {let (#strict = false)
      %EnvCheckAssign(%context, "x", 1., #strict);
      %EnvCheckAssign(%context, "y", 2., #strict);
      %context["x" ,
               {[#proto: null,
                 #class: "Object",
                 #extensible: true]}]}}
\end{verbatim}
Note the functions \emph{\%EnvCheckAssign} and
\emph{\%defineGlobalVar}. These are a part of $\lambda$S5's initial
environment, are inserted during desugaring, and exist in the initial
context's environment and store courtesy of
\emph{inject}. \emph{\%EnvCheckAssign} determines whether a particular
variable exists as a property in the \texttt{\%context}, throwing the
appropriate \texttt{ReferenceError} if the check
fails. \emph{\%defineGlobalVar} does precisely what the name purports:
it defines a global variable by inserting it as a property into
\texttt{\%context}.

With a monovariant analysis (\'{a} la 0CFA), the arguments of each
call to these integral functions will conflate, wreaking havok on
precision. In our implementation, we overcome this obstacle with
aggressive inlining of a set of library functions, including the two
above. With unique source positions (i.e. unique store addresses),
precision is regained. The above example expands from 10 to nearly
4,000 lines of $\lambda$S5. A more elegant solution would involve
variant context-sensitivity, allowing for perfect precision in
library functions.

\subsection{Objects}

JavaScript's objects are by far the most complex aspect of the
language. Objects operates as maps, where properties can be accessed,
set, and deleted both statically and dynamically. Previous work on
object maps~\cite{liang2012hash, fulara2012generic} exposed the core
problem for abstractions of indexed containers: when a genuine partial
order (as opposed to a flat partial order) occurs in the domain of
abstract keys, soundness induces monotonic update. Any abstraction of
indexed containers with genuine order in its keys' abstract domain
faces this fundamental problem.

Consider the following example with a basic key-value map: if we want
to update a key abstracted using prefix-suffix domains (which
abstracts strings by their longest common prefix and
suffix)~\cite{costantini2011static}---for instance, (foo, .cc) to
{3}---it is not enough to merely ``insert'' the entry into the
abstract map. Rather, to preserve monotonicity, we must also join the
entry with the values of keys, such as in (foooo, .cc), since (foooo,
.cc) $\sqsubseteq$ (foo, .cc).

\subsubsection{Intensional abstraction as a monotonic map}
A direct abstraction of indexed containers (as maps) yields a
monotonic map. Since the set of keys ($\hat\mkeys$) has its own
partial order, the abstraction must be monotonic:
\begin{align*}
\hat h \in \hat\mhashs = \hat\mkeys \xrightarrow{mon} \hat\mvals
\end{align*}
Recall that a function \emph{f} is \emph{monotonic} iff
\begin{align*}
x \sqsubseteq y\ \text{implies}\ f(x) \sqsubseteq f(y)
\end{align*}
and join on monotonic functions is point-wise:
\begin{align*}
(f \sqcup g)(x)\ =\ f(x) \sqcup g(x).
\end{align*}
The space operator $\xrightarrow{mon}$ soundly captures the behavior
of abstract containers.

\subsubsection{Extensional representation}
The direct structural abstraction of this representation is a set of
abstract key-value pairs:
\begin{align*}
\hat h' \in \hat \mhashs ' = \mpset (\hat\mkeys \times \hat\mvals).
\end{align*}
where both the keys $\hat\mkeys$\ and values $\hat\mvals$\  are defined over some
specific abstract domains, such as the interval domain for integers,
and the character inclusion domain or prefix-suffix
domain~\cite{costantini2011static} for strings, et
cetera. Particularly, we define the bottom of an extensional abstract
indexed container as $\hat h_\perp$, which meanns there is no entry in
the container.

\subsubsection{Why \emph{monotonic} maps are problematic}
Attempting to use the standard balanced-tree- or hash-backed
implementation of maps for a monotonic map is costly, and in some
cases, even impossible. While look-ups remain logarithmic and
efficient, updates (joins) become extremely expensive—even exponential
for some abstract domains for keys. The problem is that when the
mapping $x \mapsto y$ is inserted, all keys weaker than $x$ must also
map to a value consistent with $y$. Finitely enumerating all values
weaker than $x$ can be expensive (or impossible).

We can formally represent this process by defining an update operator,
$\mmonjoin$, on non-monotonic maps that keeps them consistent with
the monotonic join:
\begin{align*}
f \mmonjoin [x \mapsto y]\ =\ f \sqcup \bigsqcup_{\substack{\perp
    \sqsubset x' \sqsubseteq x}} [x' \mapsto y].
\end{align*}
That is, if $f:X \rightarrow Y$ and $g:X \xrightarrow{mon} Y$ and
$f\ =\ g$, then:
\begin{align*}
f \mmonjoin [x \mapsto y]\ =\ g \sqcup [x \mapsto y].
\end{align*}
Implemented literally, this provides a way to use standard
abstractions of non-monotonic maps to implement monotonic
maps---unless the abstract key domain has unbounded height. In that
case, there may not even be a finite extensional representation of a
monotonic map as a non-monotonic map. Even in cases where the height
of the abstract domain is bounded (even simple abstract domains like
the character-inclusion domain for strings) yield an exponential
number of updates (exponential in the number of characters in the key
for the inclusion domain) into the non-monotonic map.

\subsubsection{Widening to \emph{keywise-disjoint} monotonic maps}
In response to the problems with extending standard abstractions for
maps to monotonic maps, we need a widening strategy for extensional
abstractions of indexed containers. This widening strategy keeps
monotonic maps in an canonical state in which more efficient update
and look-up operations are possible.

The naive approach depends on finding keys subsumed by the key
supplied for updates and look-ups, yet it is costly, depending on the
abstract domain. Instead of enumerating all weaker keys, we perform
restructuring on the extensional abstract indexed container whenever
updating the map. This process leaves us with \emph{keywise-disjoint}
monotonic maps.

First, a few definitions. The interleaving relation:

\begin{mdef}
\begin{align*}
R_{iv} \subseteq \hat\mkeys \times \hat\mkeys := \hat k\ R_{iv}\ \hat
k' \Leftrightarrow  \gamma(\hat k) \cap \gamma(\hat k') \neq \perp
\end{align*}
\end{mdef}

and the compatibility relation:

\begin{mdef}
\begin{align*}
R_{c} \subseteq \hat\mkeys \times \hat\mkeys := \hat k\ R_{c}\ \hat
k' \Leftrightarrow \hat k \sqsubseteq \hat k'\ or\ \hat k' \sqsubseteq
\hat k.
\end{align*}
\end{mdef}

\begin{mlem}
\begin{align*}
(\hat k\ R_c\ \hat k'\ \neq\ \perp\ or\ \ \hat k \cap \hat
k'\ \neq\ \perp)\ \Rightarrow\ \hat k\ R_{iv}\ \hat k'
\end{align*}
\end{mlem}

\begin{proof}
By definition 2 and the definition of $\sqcap$ on
$\hat\mkeys$, where $\hat k,\hat k' \in\ \hat\mkeys$.
\end{proof}

The key observation is that the computation of the interleaving
relation is much less expensive than actually enumerating all the
weaker keys.

\begin{mdef}
An extensional map $\hat h$ is keywise-disjoint iff for any two
entries $(\hat k_1, \hat v_1), (\hat k_2, \hat v_2) \in \hat h$, it is
not the case that $\hat k_1\ R_{iv}\ \hat k_2$.
\end{mdef}

We define an empty extensional indexed map to be keywise-disjoint.

Whenever updating a new entry $(\hat k,\hat v)$, the keywise-disjoint
property may be destroyed. We need a method to preserve this property.

To do so, we need a partial choice function, 
\begin{math}
split : \hat\mhashs
\times (\hat\mkeys \times \hat\mvals) \rightharpoonup \hat\mhashs
\times (\hat\mkeys \times \hat\mvals)
\end{math},
which removes one entry from a map whose key are interleaving with the
key in the updating pair. Specially, if $\hat h$ is empty, then the
updating entry will be inserted into $\hat h$ and the result will be
returned.

Then, we must define a
\begin{math}
compact : \hat\mhashs \times (\hat\mkeys \times \hat\mvals) \rightarrow \hat\mhashs
\end{math}
operation on maps:
\mtodo{Fix formatting:}
\begin{align*}
compact(\hat h, (\hat k, \hat v)) =
\begin{cases}
\hat h & split(\hat h, (\hat k, \hat v))\ \text{is not defined} \\
\hat h' \cup \{(\hat k_1 \sqcup \hat k, \hat v_1 \sqcup \hat v)\} &
(\hat h', (\hat k_1, \hat v_1)) = split(\hat h, (\hat k, \hat v))
\end{cases}
\end{align*}

A $crush$ operation produces a fully keywise-disjoint map by
iterating $compact$:
\begin{align*}
crush(\hat h, (\hat k, \hat v)) = compact^*(\hat h, (\hat k, \hat v))
\end{align*}
where $f^*$ means iterated application of $f$ until a fixpoint is
reached.

Informally speaking, the $crush$ function will join the entries that
are interleaving with the updating entry, preserving non-interleaving
entries after the splitting process.

\begin{mtheo}
Let $\hat h$ be a keywise-disjoint map. If joining $(\hat k, \hat v)$
into $\hat h$ using the crush function, the resulting map is still a
keywise-disjoint map.
\end{mtheo}

\begin{proof}
By the definitions of $splitting$, $compact$, and $crush$.
\end{proof}

The lookup function is straightforward:
\begin{align*}
lookup(\hat k, \hat h) = \bigsqcup_{\substack{(\hat k',\hat v)\in \hat
h \\ \hat k\ R_{iv}\ \hat k'}} \hat v
\end{align*}

\subsubsection{Examples}
\paragraph{Arrays:}
Consider an array in the example has interval domain for abstract keys
and abstract values. Suppose
\begin{align*}
arr = ([0,3], [-2,1]), ([5,7], [4,4]), ([9,11], [2,7])
\end{align*}
and we want to update the abstract key as [2,5] to
[8,8]. We apply the $crush$ function (with process flattened
out):
\begin{align*}
{([0, 3]\sqcup[2, 5]\sqcup[5, 7], [−2, 1]\sqcup[4, 4]\sqcup[8, 8]), ([9, 11], [2, 7])}
\end{align*}
So the final structure is {([0, 7], [−2,
    8]), ([9, 11], [2, 7])}.

\paragraph{Maps:}
Consider a string-keyed map, with character inclusion (where a string
is represented as a pair of must-have characters and may-have
characters, respectively) as the string abstraction:
\begin{align*}
\hat h = \{(\{\text{a}\}, \{\text{a}, \text{c}, 0\}) \mapsto \{100\}, (\{\text{b}\}, \{\text{b}, 0\}) \mapsto \{200\}\}.
\end{align*}
and we want to update the key (\{a, c\}, \{a, c\}) with \{300\}. The restructuring process is as follows:
\begin{align*}
\begin{array}{rcl}
\hat h & = & \text{\{(\{a\},\{a,c,0\})} \sqcup
\text{(\{a,c\},\{a,c\})} \mapsto \text{\{100\}} \cup
\text{\{300\},}\\
&&\ \text{(\{b\},\{b,0\})} \mapsto \text{\{200\}\}} \\
&=& \text{\{(\{a\}} \cap \text{\{a,c\},\{a,c,0\}} \cup \text{\{a,c\})}
\mapsto \text{\{100,300\},} \\
&&\ \text{(\{b,\},\{b,0\})} \mapsto \text{\{200\}\}}
\end{array}
\end{align*}

The final result is
\begin{align*}
\text{\{(\{a\}, \{a,c,0\})} \mapsto \text{\{100,300\},
  (\{b\},\{b,0\})} \mapsto \text{\{200\}}.
\end{align*}


\subsubsection{Complexity}

\mtodo{Revisit, doubts about splitting}

\subsubsection{Back to JavaScript}
Objects in JavaScript are not simple key-value maps: they map keys to
\emph{properties}. With the latest standard, JavaScript supports both
object and property descriptors, which allow more fine-tuned control
over their access.

There are two kinds of properties (as can be seen in
figure~\ref{fig:syntax}): data and accessors. As a reminder:
\mtodo{Insert prop values from machine}

Property descriptors all have both a {\bf{configuration}} key, determining
whether the property's descriptor may be modified and whether the property may
be deleted, and an {\bf{enumeration}} key, determining whether the property is
visible when its object's properties are enumerated. Data properties have a
{\bf{value}} key which holds the value associated with the property and a
{\bf{writable}} key which determines whether that value may be
modified. Accessors have {\bf{get}} and {\bf{set}} keys, which map to functions
applied when the property is accessed or updated, respectively. Abstraction of
these descriptors is lifts pointwise over the values bound by these keys; since
the keys are concrete and known, we (thankfully) do not need to treat
descriptors as a monotonic map. The values of these descriptors, like the
properties themselves, can be updated long after object initialization.

Objects each have their own set of attributes: {\bf{class}},
{\bf{code}}, {\bf{extensible}}, {\bf{primval}}, and {\bf{proto}}. Like
the descriptors of properties, object attributes can be thought of as
a non-monotonic map from these concrete keys to abstract values. The
{\bf{class}} property ascribes a nominal type to the object,
e.g. ``Array'', ``Function'', or ``Object'', which plays a role in
calls to JavaScript's {\bf{typeof}} operator, among others. All of
JavaScript's functions are objects, and the {\bf{code}} attribute maps
to a closure if the object in question is one of these function
objects. Neither of these two attributes cannot be modified after
object initialization, matching JavaScript's semantics.

The {\bf{extensible}} descriptor determines whether fields can be
added to the object, and can be modified when JavaScript's
\texttt{Object}.$freeze(\dots)$ or \texttt{Object}.$seal(\dots)$
functions are called. JavaScript's constructors that wrap primitive
values (e.g. \texttt{Number}$(\dots)$ and \texttt{Boolean}$(\dots)$)
store that wrapped value in the {\bf{primval}} attribute. The
{\bf{proto}} attribute holds a reference to the object's
prototype. Each of these three latter attributes may be modified,
matching JavaScript's semantics.

%- eval
%  - four kinds
%  - love the desugar function
%  - implementation detail: AU map

\begin{comment}
\begin{figure}
\small
\[
\begin{array}{rcl}
%\multicolumn{3}{l}{\text{Values}}\\
\mtodo{Transliterate the ADTs}
\end{array}
\]
\caption{Abstract machine components}
\label{fig:amc}
\end{figure}

\begin{figure}
\small
\[
\begin{array}{rcl}
\mtodo{Transliterate step.ml}
\end{array}
\]
\caption{Abstracted abstract machine}
\label{fig:aam}
\end{figure}
\end{comment}

\section{Results}

\section{Related Work}

%\cite{Guha:2009:USA:1526709.1526785}
%- built on $\lambda$JS, lacks 5.1 specifics and full eval
%- use string shape analysis on the assumption of JSON deserialization
%for eval
%\cite{DBLP:journals/corr/abs-1109-4467}
%- build on $\lambda$JS, lacks 5.1 specifics and eval
%for strong updates
%\cite{heidegger2009recency}

\section{Future Work}

%% what should I leave out?

%- specialized desugarer to avoid worst case for all
%- variant k cfa (perfect precision for library functions, because
%  inlining is sloppy)
%- aggro-term with GC
%- empirical testing of various lattices (plug plug-and-play)

\section{Conclusion}

% wat?

\paragraph{Acknowledgments}

% Ack!

\balance

\bibliographystyle{plainnat}
\bibliography{refs}

\end{document}
